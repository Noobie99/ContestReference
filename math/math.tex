\section{Mathe}

\begin{algorithm}{Zykel Erkennung}
	\begin{methods}
		\method{cycleDetection}{findet Zyklus von $x_0$ und länge in $f$}{b+l}
	\end{methods}
	\sourcecode{math/cycleDetection.cpp}
\end{algorithm}

\begin{algorithm}{Permutationen}
	\begin{methods}
		\method{kthperm}{findet $k$-te Permutation \big($k \in [0, n!$)\big)}{n\*\log(n)}
	\end{methods}
	\sourcecode{math/kthperm.cpp}
	\begin{methods}
		\method{permIndex}{bestimmt index der Permutation \big($\mathit{res} \in [0, n!$)\big)}{n\*\log(n)}
	\end{methods}
	\sourcecode{math/permIndex.cpp}
\end{algorithm}

\paragraph{Lemma von \textsc{Bézout}}
Sei $(x, y)$ eine Lösung der diophantischen Gleichung $ax + by = d$.
Dann lassen sich wie folgt alle Lösungen berechnen:
\[
\left(x + k\frac{b}{\ggT(a, b)},~y - k\frac{a}{\ggT(a, b)}\right)
\]

\paragraph{\textsc{Pell}-Gleichungen}
Sei $(\overline{x}, \overline{y})$ die Lösung von $x^2 - ny^2 = 1$, die $x>1$ minimiert.
Sei $(\tilde{x}, \tilde{y})$ die Lösung von $x^2-ny^2 = c$, die $x>1$ minimiert. Dann lassen
sich alle Lösungen von $x^2-ny^2=c$ berechnen durch:
\begin{align*}
	x_1&\coloneqq \tilde{x}, & y_1&\coloneqq\tilde{y}\\
	x_{k+1}&\coloneqq \overline{x}x_k+n\overline{y}y_k, & y_{k+1}&\coloneqq\overline{x}y_k+\overline{y}x_k
\end{align*}

\begin{algorithm}{ggT, kgV, erweiterter euklidischer Algorithmus}
	\runtime{\log(a) + \log(b)}
	\sourcecode{math/gcd-lcm.cpp}
	\sourcecode{math/extendedEuclid.cpp}
\end{algorithm}

\subsection{Multiplikatives Inverses von $\boldsymbol{n}$ in $\boldsymbol{\mathbb{Z}/p\mathbb{Z}}$}
\textbf{Falls $\boldsymbol{p}$ prim:}\quad $x^{-1} \equiv x^{p-2} \bmod p$

\textbf{Falls $\boldsymbol{\ggT(n, p) = 1}$:}
\begin{itemize}
	\item Erweiterter euklidischer Algorithmus liefert $\alpha$ und $\beta$ mit
	$\alpha n + \beta p = 1$.
	\item Nach Kongruenz gilt $\alpha n + \beta p \equiv \alpha n \equiv 1 \bmod p$.
	\item $n^{-1} :\equiv \alpha \bmod p$
	\end{itemize}
\textbf{Sonst $\boldsymbol{\ggT(n, p) > 1}$:}\quad Es existiert kein $x^{-1}$.
\sourcecode{math/multInv.cpp}

\begin{algorithm}{Lineare Kongruenz}
	\begin{itemize}
		\item Löst $ax\equiv b\pmod{m}$.
		\item Weitere Lösungen unterscheiden sich um \raisebox{2pt}{$\frac{m}{g}$}, es gibt
		also $g$ Lösungen modulo $m$.
	\end{itemize}
	\sourcecode{math/linearCongruence.cpp}
\end{algorithm}

\subsection{Mod-Exponent und Multiplikation über $\boldsymbol{\mathbb{F}_p}$}
%\vspace{-1.25em}
%\begin{multicols}{2}
	\method{mulMod}{berechnet $a \cdot b \bmod n$}{\log(b)}
	\sourcecode{math/modMulIterativ.cpp}
%	\vfill\null\columnbreak
	\method{powMod}{berechnet $a^b \bmod n$}{\log(b)}
	\sourcecode{math/modPowIterativ.cpp}
%\end{multicols}
%\vspace{-2.75em}
\begin{itemize}
	\item für $a > 10^9$ \code{__int128} oder \code{modMul} benutzten! 
\end{itemize}

\begin{algorithm}{Matrix-Exponent}
	\begin{methods}
		\method{precalc}{berechnet $m^{2^b}$ vor}{\log(b)\*n^3}
		\method{calc}{berechnet $m^b_{y,x}$}{\log(b)\cdot n^2}
	\end{methods}
	\sourcecode{math/matrixPower.cpp}
\end{algorithm}

\begin{algorithm}{Berlekamp-Massey}
	\begin{methods}
		\method{BerlekampMassey}{Berechnet ein lineare Recurenz $n$-ter Ordnung}{n^2}
		\method{}{aus den ersten $2n$ Werte}{}
	\end{methods}
	\sourcecode{math/berlekampMassey.cpp}
\end{algorithm}

\begin{algorithm}{Lineare-Recurenz}
	Sei $f(n)=c_{n-1}f(n-1)+c_{n-2}f(n-2)+\dots + c_0f(0)$ eine Lineare Recurenz. Dann kann das \mbox{$k$-te} folgenglid in \runtime{n^3\log(k)} brechnet werden.
	$$\renewcommand\arraystretch{1.5}
	\setlength\arraycolsep{3pt}
	\begin{pmatrix} 
		c_{n-1} & c_{n-2} & \smash{\cdots} & \smash{\cdots} & c_0 \\
		1 & 0 & \smash{\cdots} & \smash{\cdots} & 0 \\
		0 & \smash{\ddots} & \smash{\ddots} & & \smash{\vdots} \\
		0 & \smash{\ddots} & \smash{\ddots} & \smash{\ddots} & \smash{\vdots} \\
		0 & \smash{\cdots} & 0 & 1 & 0 \\
	\end{pmatrix}^k
	\times~~
	\begin{pmatrix} 
		f(n-1) \\
		f(n-2) \\
		\smash{\vdots} \\
		\smash{\vdots} \\
		f(0) \\
	\end{pmatrix}
	~~=~~
	\begin{pmatrix} 
	f(n-1+k) \\
	f(n-2+k) \\
	\smash{\vdots} \\
	\smash{\vdots} \\
	f(k) \makebox[0pt][l]{\hspace{15pt}$\vcenter{\hbox{\huge$\leftarrow$}}$}\\
	\end{pmatrix}
	$$
\end{algorithm}

\begin{algorithm}{Chinesischer Restsatz}
	\begin{itemize}
		\item Extrem anfällig gegen Overflows. Evtl. häufig 128-Bit Integer verwenden.
		\item Direkte Formel für zwei Kongruenzen $x \equiv a \bmod n$, $x \equiv b \bmod m$:
		\[
			x \equiv a - y \cdot n \cdot \frac{a - b}{d} \bmod \frac{mn}{d}
			\qquad \text{mit} \qquad
			d := \ggT(n, m) = yn + zm
		\]
		Formel kann auch für nicht teilerfremde Moduli verwendet werden.
		Sind die Moduli nicht teilerfremd, existiert genau dann eine Lösung,
		wenn $a\equiv~b \bmod \ggT(m, n)$.
		In diesem Fall sind keine Faktoren
		auf der linken Seite erlaubt.
	\end{itemize}
	\sourcecode{math/chineseRemainder.cpp}
\end{algorithm}

\begin{algorithm}{Primzahltest \& Faktorisierung}
	\begin{itemize}
		\item für $n > 10^9$ \texttt{BigInteger} benutzten order \texttt{modMul}! 
	\end{itemize}
	\method{isPrime}{prüft ob Zahl prim ist}{\log(n)^2}
	\sourcecode{math/millerRabin.cpp}
	\method{rho}{findet zufälligen Teiler}{\sqrt[\leftroot{3}\uproot{2}3]{n}}
	\sourcecode{math/rho.cpp}
	\method{squfof}{findet zufälligen Teiler}{\sqrt[\leftroot{4}\uproot{2}4]{n}}
	\sourcecode{math/squfof.cpp}
\end{algorithm}

\begin{algorithm}{Primzahlsieb von \textsc{Eratosthenes}}
	\begin{itemize}
		\item Kann erweitert werden: Für jede Zahl den \{kleinsten, größten\} Primfaktor speichern, Liste von Primzahlen berechnen
		\item Bis $10^8$ in unter 64MB Speicher (lange Berechnung)
	\end{itemize}
	\begin{methods}
		\method{primeSieve}{berechnet Primzahlen und Anzahl}{n\*\log(\log(n))}
		\method{isPrime}{prüft ob Zahl prim ist}{1}
	\end{methods}
	\sourcecode{math/primeSieve.cpp}
\end{algorithm}

\begin{algorithm}{Teiler}
	\begin{methods}
		\method{countDivisors}{Zählt teiler von $n$}{n^\frac{1}{3}}
		\method{isPrime}{prüft ob Zahl prim ist}{1}
	\end{methods}
	\sourcecode{math/divisors.cpp}
\end{algorithm}

\subsection{Primzahlzählfunktion $\boldsymbol{\pi}$}
\begin{methods}
	\method{init}{berechnet $\pi$ bis $N$}{N\*\log(\log(N))}
	\method{phi}{zählt zu $p_i$ teilerfremde Zahlen $\leq n$ für alle $i \leq k$}{???}
	\method{pi}{zählt Primzahlen  $\leq n$ ($n < N^2$)}{n^{2/3}}
\end{methods}
\sourcecode{math/piLehmer.cpp}

\subsection{\textsc{Euler}sche $\boldsymbol{\varphi}$-Funktion}
\begin{itemize}
	\item Zählt die relativ primen Zahlen $\leq n$.

	\item Multiplikativ:
	$\gcd(a,b) = 1 \Longrightarrow \varphi(a) \cdot \varphi(b) = \varphi(ab)$

	\item $p$ prim, $k \in \mathbb{N}$:
	$~\varphi(p^k) = p^k - p^{k - 1}$

	\item \textbf{\textsc{Euler}'s Theorem:}
		Für $b \geq \varphi(c)$ gilt: $a^b \equiv a^{b \bmod \varphi(c) + \varphi(c)} \pmod{c}$. Darüber hinaus gilt: $\gcd(a, c) = 1 \Leftrightarrow a^b \equiv a^{b \bmod \varphi(c)} \pmod{c}$.
	Falls $m$ prim ist, liefert das den \textbf{kleinen Satz von \textsc{Fermat}}:
	$a^{m} \equiv a \pmod{m}$
\end{itemize}
\sourcecode{math/phi.cpp}

\begin{algorithm}{\textsc{Möbius}-Funktion und \textsc{Möbius}-Inversion}
	\begin{itemize}
		\item Seien $f,g : \mathbb{N} \to \mathbb{N}$ und  $g(n) := \sum_{d \vert n}f(d)$.
		Dann ist $f(n) = \sum_{d \vert n}g(d)\mu(\frac{n}{d})$.
		\item $\sum\limits_{d \vert n}\mu(d) =
		\begin{cases*}
		1 & falls $n = 1$\\
		0 & sonst
		\end{cases*}$
	\end{itemize}
	\textbf{Beispiel Inklusion/Exklusion:}
	Gegeben sein eine Sequenz $A={a_1,\ldots,a_n}$ von Zahlen, $1 \leq a_i \leq N$. Zähle die Anzahl der \emph{coprime subsequences}.\newline
	\textbf{Lösung}:
	Für jedes $x$, sei $cnt[x]$ die Anzahl der Vielfachen von $x$ in $A$.
	Es gibt $2^{cnt[x]}-1$ nicht leere Subsequences in $A$, die nur Vielfache von $x$ enthalten.
	Die Anzahl der Subsequences mit $\ggT=1$ ist gegeben durch $\sum_{i = 1}^N \mu(i) \cdot (2^{cnt[i]} - 1)$.
	\sourcecode{math/mobius.cpp}
\end{algorithm}

\begin{algorithm}{Primitivwurzeln}
	\begin{itemize}
		\item Primitivwurzel modulo $n$ existiert $\Leftrightarrow$ $n \in \{2,\ 4,\ p^\alpha,\ 2\cdot p^\alpha \mid\ 2 < p \in \mathbb{P},\ \alpha \in \mathbb{N}\}$
		\item es existiert entweder keine oder $\varphi(\varphi(n))$ inkongruente Primitivwurzeln
		\item Sei $g$ Primitivwurzel modulo $n$.
		Dann gilt:\newline
		Das kleinste $k$, sodass $g^k \equiv 1 \bmod n$, ist $k = \varphi(n)$.
	\end{itemize}
	\begin{methods}
		\method{isPrimitive}{prüft ob $g$ eine Primitivwurzel ist}{\log(\varphi(n))\*\log(n)}
		\method{findPrimitive}{findet Primitivwurzel (oder -1)}{\abs{ans}\*\log(\varphi(n))\*\log(n)}
	\end{methods}
	\sourcecode{math/primitiveRoot.cpp}
\end{algorithm}

\begin{algorithm}{Diskreter Logarithmus}
	\begin{methods}
		\method{solve}{bestimmt Lösung $x$ für $a^x=b \bmod m$}{\sqrt{m}\*\log(m)}
	\end{methods}
	\sourcecode{math/discreteLogarithm.cpp}
\end{algorithm}
%TODO
\begin{algorithm}{Diskrete \textrm{\textit{n}}-te Wurzel}
	\begin{methods}
		\method{root}{bestimmt Lösung $x$ für $x^a=b \bmod m$ }{\sqrt{m}\*\log(m)}
	\end{methods}
	Alle Lösungen haben die Form $g^{c + \frac{i \cdot \phi(n)}{\gcd(a, \phi(n))}}$
	\sourcecode{math/discreteNthRoot.cpp}
\end{algorithm}

\subsection{LGS über $\boldsymbol{\mathbb{F}_p}$}
\method{gauss}{löst LGS}{n^3}
\sourcecode{math/lgsFp.cpp}

\subsection{LGS über $\boldsymbol{\mathbb{R}}$}
\sourcecode{math/gauss.cpp}

\begin{algorithm}{Polynome, FFT, NTT \& andere Transformationen}
	Multipliziert Polynome $A$ und $B$.
	\begin{itemize}
		\item $\deg(A \cdot B) = \deg(A) + \deg(B)$
		\item Vektoren \lstinline{a} und \lstinline{b} müssen mindestens Größe
		$\deg(A \cdot B) + 1$ haben.
		Größe muss eine Zweierpotenz sein.
		\item Für ganzzahlige Koeffizienten: \lstinline{(int)round(real(a[i]))}
		\item xor, or und and Transform funktioniert auch mit \code{double} oder modulo einer Primzahl $p$ falls $p \geq 2^{\texttt{bits}}$
	\end{itemize}
	%\lstinputlisting{math/fft.cpp}
	%\lstinputlisting{math/ntt.cpp}
	%\textcolor{safeOrange}{$\blacksquare$} NTT code, %\textcolor{safeGreen}{$\blacksquare$} FFT code
	\sourcecode{math/transforms/all.cpp}
	Für sehr viele transforms kann die Vertauschung vorberechnet werden:
	\sourcecode{math/transforms/fftPerm.cpp}
	Multiplikation mit 2 transforms statt 3: (nur benutzten wenn nötig!)
	\sourcecode{math/transforms/fftMul.cpp}
\end{algorithm}

\begin{algorithm}{Numerisch Extremstelle bestimmen}
	\sourcecode{math/goldenSectionSearch.cpp}
\end{algorithm}

\begin{algorithm}{Numerisch Integrieren, Simpsonregel}
	\sourcecode{math/simpson.cpp}
\end{algorithm}

\begin{algorithm}{Longest Increasing Subsequence}
	\begin{itemize}
		\item \code{lower\_bound} $\Rightarrow$ streng monoton
		\item \code{upper\_bound} $\Rightarrow$ monoton
	\end{itemize}
	\sourcecode{math/longestIncreasingSubsequence.cpp}
\end{algorithm}

\begin{algorithm}{\textsc{Legendre}-Symbol}
	Sei $p \geq 3$ eine Primzahl, $a \in \mathbb{Z}$:
	\begin{align*}
		\legendre{a}{p} &=
		\begin{cases*}
		\hphantom{-}0 & falls $p~\vert~a$ \\[-1ex]
		\hphantom{-}1 & falls $\exists x \in \mathbb{Z}\backslash p\mathbb{Z} : a \equiv x^2 \bmod p$ \\[-1ex]
		-1 & sonst
		\end{cases*} \\
		\legendre{-1}{p} = (-1)^{\frac{p - 1}{2}} &=
		\begin{cases*}
		\hphantom{-}1 & falls $p \equiv 1 \bmod 4$ \\[-1ex]
		-1 & falls $p \equiv 3 \bmod 4$
		\end{cases*} \\
		\legendre{2}{p} = (-1)^{\frac{p^2 - 1}{8}} &=
		\begin{cases*}
		\hphantom{-}1 & falls $p \equiv \pm 1 \bmod 8$ \\[-1ex]
		-1 & falls $p \equiv \pm 3 \bmod 8$
		\end{cases*}
	\end{align*}
	\begin{align*}
		\legendre{p}{q} \cdot \legendre{q}{p} = (-1)^{\frac{p - 1}{2} \cdot \frac{q - 1}{2}} &&
		\legendre{a}{p} \equiv a^{\frac{p-1}{2}}\bmod p
	\end{align*}
	\sourcecode{math/legendre.cpp}
\end{algorithm}

\begin{algorithm}{Inversionszahl}
	\sourcecode{math/inversions.cpp}
\end{algorithm}

\subsection{Satz von \textsc{Sprague-Grundy}}
Weise jedem Zustand $X$ wie folgt eine \textsc{Grundy}-Zahl $g\left(X\right)$ zu:
\[
g\left(X\right) := \min\left\{
\mathbb{Z}_0^+ \setminus
\left\{g\left(Y\right) \mid Y \text{ von } X \text{ aus direkt erreichbar}\right\}
\right\} 
\]
$X$ ist genau dann gewonnen, wenn $g\left(X\right) > 0$ ist.\\
Wenn man $k$ Spiele in den Zuständen $X_1, \ldots, X_k$ hat, dann ist die \textsc{Grundy}-Zahl des Gesamtzustandes $g\left(X_1\right) \oplus \ldots \oplus g\left(X_k\right)$.

\subsection{Kombinatorik}

\paragraph{Wilsons Theorem}
A number $n$ is prime if and only if 
$(n-1)!\equiv -1\bmod{n}$.\\
($n$ is prime if and only if $(m-1)!\cdot(n-m)!\equiv(-1)^m\bmod{n}$ for all $m$ in $\{1,\dots,n\}$)
\begin{align*}
	(n-1)!\equiv\begin{cases}
		-1\bmod{n},&\mathrm{falls}~n \in \mathbb{P}\\
		\hphantom{-}2\bmod{n},&\mathrm{falls}~n = 4\\
		\hphantom{-}0\bmod{n},&\mathrm{sonst}
	\end{cases}
\end{align*}

\paragraph{\textsc{Zeckendorfs} Theorem}
Jede positive natürliche Zahl kann eindeutig als Summe einer oder mehrerer
verschiedener \textsc{Fibonacci}-Zahlen geschrieben werden, sodass keine zwei
aufeinanderfolgenden \textsc{Fibonacci}-Zahlen in der Summe vorkommen.\\
\emph{Lösung:} Greedy, nimm immer die größte \textsc{Fibonacci}-Zahl, die noch
hineinpasst.

\paragraph{\textsc{Lucas}-Theorem}
Ist $p$ prim, $m=\sum_{i=0}^km_ip^i$, $n=\sum_{i=0}^kn_ip^i$ ($p$-adische Darstellung),
so gilt
\vspace{-0.75\baselineskip}
\[
	\binom{m}{n} \equiv \prod_{i=0}^k\binom{m_i}{n_i} \bmod{p}.
\]

\subsection{The Twelvefold Way \textnormal{(verteile $n$ Bälle auf $k$ Boxen)}}
\begin{expandtable}
\begin{tabularx}{\linewidth}{|C|CICICIC|}
	\hline
	Bälle & identisch & verschieden & identisch      & verschieden \\
	Boxen & identisch & identisch      & verschieden & verschieden \\
	\hline
	-- &
	$p_k(n + k)$ &
	$\sum\limits_{i = 0}^k \stirlingII{n}{i}$ &
	$\binom{n + k - 1}{k - 1}$ &
	$k^n$ \\
	\grayhline

	\makecell{Bälle pro\\Box $\geq 1$} &
	$p_k(n)$ &
	$\stirlingII{n}{k}$ &
	$\binom{n - 1}{k - 1}$ &
	$k! \stirlingII{n}{k}$ \\
	\grayhline

	\makecell{Bälle pro\\Box $\leq 1$} &
	$[n \leq k]$ &
	$[n \leq k]$ &
	$\binom{k}{n}$ &
	$n! \binom{k}{n}$ \\
	\hline
	\multicolumn{5}{|l|}{
		$[\text{Bedingung}]$: \lstinline{return Bedingung ? 1 : 0;}
	} \\
	\hline
\end{tabularx}
\end{expandtable}


\paragraph{\textsc{Catalan}-Zahlen}
\begin{itemize}
	\item Die \textsc{Catalan}-Zahl $C_n$ gibt an:
	\begin{itemize}
		\item Anzahl der Binärbäume mit $n$ nicht unterscheidbaren Knoten.
		\item Anzahl der validen Klammerausdrücke mit $n$ Klammerpaaren.
		\item Anzahl der korrekten Klammerungen von $n+1$ Faktoren.
		\item Anzahl der Möglichkeiten ein konvexes Polygon mit $n + 2$ Ecken in Dreiecke zu zerlegen.
		\item Anzahl der monotonen Pfade (zwischen gegenüberliegenden Ecken) in
		einem $n \times n$-Gitter, die nicht die Diagonale kreuzen.
	\end{itemize}
\end{itemize}
\[C_0 = 1\qquad C_n = \sum\limits_{k = 0}^{n - 1} C_kC_{n - 1 - k} =
\frac{1}{n + 1}\binom{2n}{n} = \frac{4n - 2}{n+1} \cdot C_{n-1}\]
\begin{itemize}
	\item Formel $1$ erlaubt berechnung ohne Division in \runtime{n^2}
	\item Formel $2$ und $3$ erlauben berechnung in \runtime{n}
\end{itemize}

\paragraph{\textsc{Euler}-Zahlen 1. Ordnung}
Die Anzahl der Permutationen von $\{1, \ldots, n\}$ mit genau $k$ Anstiegen.
Für die $n$-te Zahl gibt es $n$ mögliche Positionen zum Einfügen.
Dabei wird entweder ein Ansteig in zwei gesplitted oder ein Anstieg um $n$ ergänzt.
\[\eulerI{n}{0} = \eulerI{n}{n-1} = 1 \quad
\eulerI{n}{k} = (k+1) \eulerI{n-1}{k} + (n-k) \eulerI{n-1}{k-1}=
\sum_{i=0}^{k} (-1)^i\binom{n+1}{i}(k+1-i)^n\]
\begin{itemize}
	\item Formel $1$ erlaubt berechnung ohne Division in \runtime{n^2}
	\item Formel $2$ erlaubt berechnung in \runtime{n\log(n)}
\end{itemize}

\paragraph{\textsc{Euler}-Zahlen 2. Ordnung}
Die Anzahl der Permutationen von $\{1,1, \ldots, n,n\}$ mit genau $k$ Anstiegen.
\[\eulerII{n}{0} = 1 \qquad\eulerII{n}{n} = 0 \qquad\eulerII{n}{k} = (k+1) \eulerII{n-1}{k} + (2n-k-1) \eulerII{n-1}{k-1}\]
\begin{itemize}
	\item Formel erlaubt berechnung ohne Division in \runtime{n^2}
\end{itemize}

\paragraph{\textsc{Stirling}-Zahlen 1. Ordnung}
Die Anzahl der Permutationen von $\{1, \ldots, n\}$ mit genau $k$ Zyklen.
Es gibt zwei Möglichkeiten für die $n$-te Zahl. Entweder sie bildet einen eigene Zyklus, oder sie kann an jeder Position in jedem Zyklus einsortiert werden.
\[\stirlingI{0}{0} = 1 \qquad
\stirlingI{n}{0} = \stirlingI{0}{n} = 0 \qquad
\stirlingI{n}{k} = \stirlingI{n-1}{k-1} + (n-1) \stirlingI{n-1}{k}\]
\begin{itemize}
	\item Formel erlaubt berechnung ohne Division in \runtime{n^2}
\end{itemize}

\paragraph{\textsc{Stirling}-Zahlen 2. Ordnung}
Die Anzahl der Möglichkeiten $n$ Elemente in $k$ nichtleere Teilmengen zu zerlegen.
Es gibt $k$ Möglichkeiten die $n$ in eine $n-1$-Partition einzuordnen.
Dazu kommt der Fall, dass die $n$ in ihrer eigenen Teilmenge (alleine) steht.
\[\stirlingII{n}{1} = \stirlingII{n}{n} = 1 \qquad
\stirlingII{n}{k} = k \stirlingII{n-1}{k} + \stirlingII{n-1}{k-1} =
\frac{1}{k!} \sum\limits_{i=0}^{k} (-1)^{k-i}\binom{k}{i}i^n\]
\begin{itemize}
	\item Formel $1$ erlaubt berechnung ohne Division in \runtime{n^2}
	\item Formel $2$ erlaubt berechnung in \runtime{n\log(n)}
\end{itemize}

\paragraph{\textsc{Bell}-Zahlen}
Anzahl der Partitionen von $\{1, \ldots, n\}$.
Wie \textsc{Striling}-Zahlen 2. Ordnung ohne Limit durch $k$.
\[B_1 = 1 \qquad
B_n = \sum\limits_{k = 0}^{n - 1} B_k\binom{n-1}{k}
= \sum\limits_{k = 0}^{n}\stirlingII{n}{k}\qquad\qquad B_{p^m+n}\equiv m\cdot B_n + B_{n+1} \bmod{p}\]

\paragraph{Partitions}
Die Anzahl der Partitionen von $n$ in genau $k$ positive Summanden.
Die Anzahl der Partitionen von $n$ mit Elementen aus ${1,\dots,k}$.
\begin{align*}
	p_0(0)=1 \qquad p_k(n)&=0 \text{ für } k > n \text{ oder } n \leq 0 \text{ oder } k \leq 0\\
	p_k(n)&= p_k(n-k) + p_{k-1}(n-1)\\[2pt]
	p(n)&=\sum_{k=1}^{n} p_k(n)=p_n(2n)=\sum\limits_{k\neq0}^\infty(-1)^{k+1}p\bigg(n - \frac{k(3k-1)}{2}\bigg)
\end{align*}
\begin{itemize}
	\item in Formel $3$ kann abgebrochen werden wenn $\frac{k(3k-1)}{2} > n$.
	\item Die Anzahl der Partitionen von $n$ in bis zu $k$ positive Summanden ist $\sum\limits_{i=0}^{k}p_i(n)=p_k(n+k)$.
\end{itemize}

\paragraph{Binomialkoeffizienten}
Die Anzahl der \mbox{$k$-elementigen} Teilmengen einer \mbox{$n$-elementigen} Menge.

\textbf{WICHTIG:} Binomialkoeffizient in \runtime{1} berechnen indem man $x!$ vorberechnet.

%\begin{algorithm}{Binomialkoeffizienten}
	\begin{methods}
		\method{calc\_binom}{berechnet Binomialkoeffizient $(n \le 61)$}{k}
	\end{methods}
	\sourcecode{math/binomial.cpp}

	\begin{methods}
		\method{calc\_binom}{berechnet Binomialkoeffizient modulo Primzahl $p$}{p-n}
	\end{methods}
	\textbf{Wichtig:} $p > n$
	\sourcecode{math/binomial3.cpp}
	
	\begin{methods}
		\method{calc\_binom}{berechnet Primfaktoren vom Binomialkoeffizient}{n}
	\end{methods}
	\textbf{WICHTIG:} braucht alle Primzahlen $\leq n$
	\sourcecode{math/binomial2.cpp}
%\end{algorithm}

%\begin{expandtable}
\begin{tabularx}{\linewidth}{|l|X|}
	\hline
	\multicolumn{2}{|c|}{Berühmte Zahlen} \\
	\hline
	\textsc{Fibonacci}	&
	$f(0) = 0 \quad
	f(1) = 1 \quad
	f(n+2) = f(n+1) + f(n)$ \\
	\grayhline
	
	\textsc{Catalan}	&
	$C_0 = 1 \qquad
	C_n = \sum\limits_{k = 0}^{n - 1} C_kC_{n - 1 - k} =
	\frac{1}{n + 1}\binom{2n}{n} = \frac{2(2n - 1)}{n+1} \cdot C_{n-1}$ \\
	\grayhline
	
	\textsc{Euler} I &
	$\eulerI{n}{0} = \eulerI{n}{n-1} = 1 \qquad
	\eulerI{n}{k} = (k+1) \eulerI{n-1}{k} + (n-k) \eulerI{n-1}{k-1} $ \\
	\grayhline
	
	\textsc{Euler} II &
	$\eulerII{n}{0} = 1 \quad
	\eulerII{n}{n} = 0 \quad$\\
	& $\eulerII{n}{k} = (k+1) \eulerII{n-1}{k} + (2n-k-1) \eulerII{n-1}{k-1}$ \\
	\grayhline
	
	\textsc{Stirling} I &
	$\stirlingI{0}{0} = 1 \qquad
	\stirlingI{n}{0} = \stirlingI{0}{n} = 0 \qquad
	\stirlingI{n}{k} = \stirlingI{n-1}{k-1} + (n-1) \stirlingI{n-1}{k}$ \\
	\grayhline
	
	\textsc{Stirling} II &
	$\stirlingII{n}{1} = \stirlingII{n}{n} = 1 \qquad
	\stirlingII{n}{k} = k \stirlingII{n-1}{k} + \stirlingII{n-1}{k-1} =
	\frac{1}{k!} \sum\limits_{j=0}^{k} (-1)^{k-j}\binom{k}{j}j^n$\\
	\grayhline
	
	\textsc{Bell} &
	$B_1 = 1 \qquad
	B_n = \sum\limits_{k = 0}^{n - 1} B_k\binom{n-1}{k}
	= \sum\limits_{k = 0}^{n}\stirlingII{n}{k}$\\
	\grayhline
	
	\textsc{Partitions} &
	$p(0,0) = 1 \quad
	p(n,k) = 0 \text{ für } k > n \text{ oder } n \leq 0 \text{ oder } k \leq 0$ \\
	& $p(n,k) = p(n-k,k) + p(n-1,k-1)$\\
	\grayhline
	
	\textsc{Partitions} &
	$f(0) = 1 \quad f(n) = 0~(n < 0)$ \\
	& $f(n)=\sum\limits_{k=1}^\infty(-1)^{k-1}f(n - \frac{k(3k+1)}{2})+\sum\limits_{k=1}^\infty(-1)^{k-1}f(n - \frac{k(3k-1)}{2})$\\
	
	\hline
\end{tabularx}
\end{expandtable}


\begin{algorithm}[optional]{Big Integers}
	\sourcecode{math/bigint.cpp}
\end{algorithm}
